\chapter{Case Studies for D-CODE}

In this chapter, we record and reflect upon some of the application of this D-CODE project coding scheme to serve as a example so that we could be quick and convenient when we roll out new project coding scheme.

Here are some of the examples from our internal projects ideation.

\section{DocSky project family}

DocSky is my (author's) personal project (actually a series of projects) dedicated to develop the documentation ecosystem. Potential outcome of this project is expected to be static documentation websites, mailing list (and templates), publications management and formatting and so on.

Let's mark this very document as a coding scheme design project as \texttt{DSKY-MOD-1}, referring to \texttt{DSKY} project family, \texttt{MOD} as project module and \texttt{1} is to refer this particular coding scheme project. The reference calling alias is \texttt{D-CODE} to make it easy to remember, which is not compliant name with the coding scheme that we have here.

\section{WizzCore DevBoard series from LabTown}

We have already taken a case study on this in the previous chapter. Let's take \texttt{WZCORE-GEN-1-R} which means it is a WizzCore board for general use, first major edition, \texttt{R} is referred to use in Robots, other format to application designators can be three letter combination such as \texttt{RBT}, \texttt{IOT}, \texttt{PLC}. Other products descriptors can also be added to this.

\section{Taskmaster - random software tool}

It is not necessarily advised to use a hardcore product coding scheme for the virtual products. However, if their own deliverable has to be distinguished from one another, we can go with \texttt{TM-1-PRO} which means Taskmaster tool of first edition (major release), under \texttt{PRO} category (alternatives: \texttt{ECO}, \texttt{ENT}, \texttt{EDU}). Semantic versioning system is recommended for software related projects as opposed to having a revision number in the coding scheme. \cite{preston-werner_2013}

\clearpage