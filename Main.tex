\title{CPE241 Database - Hotel Management Project}
\RequirePackage[l2tabu]{nag}		% Warns for incorrect (obsolete) LaTeX usage
%
%
% File: main.tex
% Author: Yuil Tripathee
% Description: Contains the thesis template using memoir class,
% which is mainly based on book class but permits better control of
% chapter styles for example. This template is an adaptation and
% modification of Oscar's and Nowab Md. Aminul Haq's.
% Mod from: https://www.overleaf.com/articles/bangladesh-university-of-engineering-and-technolog-buet-thesis-template/hgspdxtdnjyp
%
% Memoir is a flexible class for typesetting poetry, fiction,
% non-fiction and mathematical works as books, reports, articles...
% http://www.ctan.org/tex-archive/macros/latex/contrib/memoir/
%
% Memoir class loads useful packages by default (see manual).
% a4paper => draft, legno => (removed, after 11pt), openbib => removed, after 11pt
\documentclass[a4paper, 12pt, oneside]{memoir}
%add 'draft' to turn draft option on (see below)
% Adding metadata:x

\usepackage{vhistory}
\usepackage{comment}
\usepackage{gensymb}
\usepackage{datetime}
\usepackage{ifpdf}
\usepackage{indentfirst}
\ifpdf
\pdfinfo{
   /Author (Yuil Tripathee)
   /Title (DocSky Whitepaper Template MK I)
   /Keywords (DocSky; Whitepaper;Template)
   /CreationDate (D:\pdfdate)
}
\fi

% When draft option is on.
\ifdraftdoc
	\usepackage{draftwatermark}				%Sets watermarks up.
	\SetWatermarkScale{0.3}
	\SetWatermarkText{\bf Draft: \today}
\fi
%
% Declare figure/table as a subfloat.
\newsubfloat{figure}
\newsubfloat{table}
% Better page layout for A4 paper, see memoir manual.
\settrimmedsize{297mm}{210mm}{*}
\setlength{\trimtop}{0pt}
\setlength{\trimedge}{\stockwidth}
\addtolength{\trimedge}{-\paperwidth}
\settypeblocksize{696.24pt}{451.44pt}{*}
\setulmargins{1in}{*}{*}
\setlrmargins{*}{*}{1}
\setmarginnotes{17pt}{51pt}{\onelineskip}
\setheadfoot{\onelineskip}{2\onelineskip}
\setheaderspaces{*}{\onelineskip}{*}
\checkandfixthelayout
%
\frenchspacing

% Font with math support: New Century Schoolbook
\usepackage{fouriernc}
\usepackage{}
\usepackage[T1]{fontenc}
% \usepackage{mathptmx}
% other usepackage fonts, recommended: lmodern, mathptmx
% \usepackage{tgschola} % \setmainfont{TeX Gyre Schola}
%

% UoB guidelines:
%
% Text should be in double or 1.5 line spacing, and font size should be
% chosen to ensure clarity and legibility for the main text and for any
% quotations and footnotes. Margins should allow for eventual hard binding.
%
% Note: This is automatically set by memoir class. Nevertheless \OnehalfSpacing
% enables double spacing but leaves single spaced for captions for instance.
\OnehalfSpacing
%
% Sets numbering division level
\setsecnumdepth{subsection}
\maxsecnumdepth{subsubsection}
%
% Chapter style (taken and slightly modified from Lars Madsen Memoir Chapter
% Styles document
\usepackage{calc,soul,fourier}
\makeatletter
\newlength\dlf@normtxtw
\setlength\dlf@normtxtw{\textwidth}
\newsavebox{\feline@chapter}
\newcommand\feline@chapter@marker[1][4cm]{%
	\sbox\feline@chapter{%
		\resizebox{!}{#1}{\fboxsep=1pt%
			\colorbox{gray}{\color{white}\thechapter}%
		}}%
		\rotatebox{90}{%
			\resizebox{%
				\heightof{\usebox{\feline@chapter}}+\depthof{\usebox{\feline@chapter}}}%
			{!}{\scshape\so\@chapapp}}\quad%
		\raisebox{\depthof{\usebox{\feline@chapter}}}{\usebox{\feline@chapter}}%
}
\newcommand\feline@chm[1][4cm]{%
	\sbox\feline@chapter{\feline@chapter@marker[#1]}%
	\makebox[0pt][c]{% aka \rlap
		\makebox[1cm][r]{\usebox\feline@chapter}%
	}}
%\makechapterstyle{daleifmodif}{
	\renewcommand\chapnamefont{\normalfont\Large\scshape\raggedleft\so}
	\renewcommand\chaptitlefont{\normalfont\Large\bfseries\scshape}
	%\renewcommand\chapternamenum{} \renewcommand\printchaptername{}
	%\renewcommand\printchapternum{\null\hfill\feline@chm[2.5cm]\par}
	\renewcommand\afterchapternum{\par\vskip\midchapskip}
	%\renewcommand\printchaptertitle[1]{\color{gray}\chaptitlefont\raggedleft ##1\par}
%}
\makeatother
\chapterstyle{daleifmodif}
%
% UoB guidelines:
%
% The pages should be numbered consecutively at the bottom centre of the
% page.
\makepagestyle{myvf}
\makeoddfoot{myvf}{}{\thepage}{}
\makeevenfoot{myvf}{}{\thepage}{}
\makeheadrule{myvf}{\textwidth}{\normalrulethickness}
\makeevenhead{myvf}{\small\textsc{\leftmark}}{}{}
\makeoddhead{myvf}{}{}{\small\textsc{\rightmark}}
\pagestyle{myvf}
%
% Oscar's command (it works):
% Fills blank pages until next odd-numbered page. Used to emulate single-sided
% frontmatter. This will work for title, abstract and declaration. Though the
% contents sections will each start on an odd-numbered page they will
% spill over onto the even-numbered pages if extending beyond one page
% (hopefully, this is ok).
\newcommand{\clearemptydoublepage}{\newpage{\thispagestyle{empty}\cleardoublepage}}
%
%
% Creates indexes for Table of Contents, List of Figures, List of Tables and Index
\makeindex
% \printglossaries below creates a list of abbreviations. \gls and related
% commands are then used throughout the text, so that latex can automatically
% keep track of which abbreviations have already been defined in the text.
%
% The import command enables each chapter tex file to use relative paths when
% accessing supplementary files. For example, to include
% chapters/brewing/images/figure1.png from chapters/brewing/brewing.tex we can
% use
% \includegraphics{images/figure1}
% instead of
% \includegraphics{chapters/brewing/images/figure1}
\usepackage{import}

% Add other packages needed for chapters here. For example:
\usepackage{lipsum}					%Needed to create dummy text
\usepackage{amsfonts} 					%Calls Amer. Math. Soc. (AMS) fonts
\usepackage[centertags]{amsmath}			%Writes maths centred down
\usepackage{stmaryrd}					%New AMS symbols
\usepackage{amssymb}					%Calls AMS symbols
\usepackage{amsthm}
\usepackage{amsmath}%Calls AMS theorem environment
\usepackage{newlfont}					%Helpful package for fonts and symbols
\usepackage{layouts}					%Layout diagrams
\usepackage{graphicx}					%Calls figure environment
\usepackage{longtable,rotating}			%Long tab environments including rotation.
\usepackage[applemac]{inputenc}			%Needed to encode non-english characters
									%directly for mac
\usepackage{colortbl}					%Makes coloured tables
\usepackage{wasysym}					%More math symbols
\usepackage{mathrsfs}					%Even more math symbols
\usepackage{float}						%Helps to place figures, tables, etc.
\usepackage{verbatim}					%Permits pre-formated text insertion
\usepackage{upgreek }					%Calls other kind of greek alphabet
\usepackage{latexsym}					%Extra symbols
\usepackage[square,numbers,
		     sort&compress]{natbib}		%Calls bibliography commands
\usepackage{url}						%Supports url commands
\usepackage{etex}						%eTeXÕs extended support for counters
\usepackage{fixltx2e}					%Eliminates some in felicities of the
									%original LaTeX kernel
\usepackage[english]{babel}		%For languages characters and hyphenation
\usepackage{color}                    				%Creates coloured text and background
\usepackage[colorlinks=true,
		     allcolors=black]{hyperref}              %Creates hyperlinks in cross references
\usepackage{memhfixc}					%Must be used on memoir document
									%class after hyperref
\usepackage{enumerate}					%For enumeration counter
\usepackage{footnote}					%For footnotes
\usepackage{microtype}					%Makes pdf look better.
\usepackage{rotfloat}					%For rotating and float environments as tables,
									%figures, etc.
\usepackage{alltt}						%LaTeX commands are not disabled in
									%verbatim-like environment
\usepackage[version=0.96]{pgf}			%PGF/TikZ is a tandem of languages for producing vector graphics from a
\usepackage{tikz}
%geometric/algebraic description.
\usetikzlibrary{arrows,shapes,snakes,
		       automata,backgrounds,
		       petri,topaths}				%To use diverse features from tikz
%
%Reduce widows  (the last line of a paragraph at the start of a page) and orphans
% (the first line of paragraph at the end of a page)
\widowpenalty=1000
\clubpenalty=1000
%
% New command definitions for my thesis
%
\newcommand{\keywords}[1]{\par\noindent{\small{\bf Keywords:} #1}} %Defines keywords small section
\newcommand{\parcial}[2]{\frac{\partial#1}{\partial#2}}                             %Defines a partial operator
\newcommand{\vectorr}[1]{\mathbf{#1}}                                                        %Defines a bold vector
\newcommand{\vecol}[2]{\left(                                                                         %Defines a column vector
	\begin{array}{c}
		\displaystyle#1 \\
		\displaystyle#2
	\end{array}\right)}
\newcommand{\mados}[4]{\left(                                                                       %Defines a 2x2 matrix
	\begin{array}{cc}
		\displaystyle#1 &\displaystyle #2 \\
		\displaystyle#3 & \displaystyle#4
	\end{array}\right)}
\newcommand{\pgftextcircled}[1]{                                                                    %Defines encircled text
    \setbox0=\hbox{#1}%
    \dimen0\wd0%
    \divide\dimen0 by 2%
    \begin{tikzpicture}[baseline=(a.base)]%
        \useasboundingbox (-\the\dimen0,0pt) rectangle (\the\dimen0,1pt);
        \node[circle,draw,outer sep=0pt,inner sep=0.1ex] (a) {#1};
    \end{tikzpicture}
}
\newcommand{\range}[1]{\textnormal{range }#1}                                             %Defines range operator
\newcommand{\innerp}[2]{\left\langle#1,#2\right\rangle}                                 %Defines inner product
\newcommand{\prom}[1]{\left\langle#1\right\rangle}                                         %Defines average operator
\newcommand{\tra}[1]{\textnormal{tra} \: #1}                                                       %Defines trace operator
\newcommand{\sign}[1]{\textnormal{sign\,}#1}                                                   %Defines sign operator
\newcommand{\sech}[1]{\textnormal{sech} #1}                                                  %Defines sech
\newcommand{\diag}[1]{\textnormal{diag} #1}                                                    %Defines diag operator
\newcommand{\arcsech}[1]{\textnormal{arcsech} #1}                                       %Defines arcsech
\newcommand{\arctanh}[1]{\textnormal{arctanh} #1}                                         %Defines arctanh
%Change tombstone symbol
\newcommand{\blackged}{\hfill$\blacksquare$}
\newcommand{\whiteged}{\hfill$\square$}
\newcounter{proofcount}
\renewenvironment{proof}[1][\proofname.]{\par
 \ifnum \theproofcount>0 \pushQED{\whiteged} \else \pushQED{\blackged} \fi%
 \refstepcounter{proofcount}
 \normalfont
 \trivlist
 \item[\hskip\labelsep
       \itshape
   {\bf\em #1}]\ignorespaces
}{%
 \addtocounter{proofcount}{-1}
 \popQED\endtrivlist
}
%
%
% New definition of square root:
% it renames \sqrt as \oldsqrt
\let\oldsqrt\sqrt
% it defines the new \sqrt in terms of the old one
\def\sqrt{\mathpalette\DHLhksqrt}
\def\DHLhksqrt#1#2{%
\setbox0=\hbox{$#1\oldsqrt{#2\,}$}\dimen0=\ht0
\advance\dimen0-0.2\ht0
\setbox2=\hbox{\vrule height\ht0 depth -\dimen0}%
{\box0\lower0.4pt\box2}}
%
% My caption style
\newcommand{\mycaption}[2][\@empty]{
	\captionnamefont{\scshape}
	\changecaptionwidth
	\captionwidth{0.9\linewidth}
	\captiondelim{.\:}
	\indentcaption{0.75cm}
	\captionstyle[\centering]{}
	\setlength{\belowcaptionskip}{10pt}
	\ifx \@empty#1 \caption{#2}\else \caption[#1]{#2}
}
%
% My subcaption style
\newcommand{\mysubcaption}[2][\@empty]{
	\subcaptionsize{\small}
	\hangsubcaption
	\subcaptionlabelfont{\rmfamily}
	\sidecapstyle{\raggedright}
	\setlength{\belowcaptionskip}{10pt}
	\ifx \@empty#1 \subcaption{#2}\else \subcaption[#1]{#2}
}

%An initial of the very first character of the content
\usepackage{lettrine}
\newcommand{\initial}[1]{%
	\lettrine[lines=3,lhang=0.33,nindent=0em]{
		\color{gray}
     		{\textsc{#1}}}{}}
%
% Theorem styles used in my thesis
%
\theoremstyle{plain}
\newtheorem{theo}{Theorem}[chapter]
\theoremstyle{plain}
\newtheorem{prop}{Proposition}[chapter]
\theoremstyle{plain}
\theoremstyle{definition}
\newtheorem{dfn}{Definition}[chapter]
\theoremstyle{plain}
\newtheorem{lema}{Lemma}[chapter]
\theoremstyle{plain}
\newtheorem{cor}{Corollary}[chapter]
\theoremstyle{plain}
\newtheorem{resu}{Result}[chapter]
%
% Hyphenation for some words
%
\hyphenation{res-pec-tively}
\hyphenation{mono-ti-ca-lly}
\hyphenation{hypo-the-sis}
\hyphenation{para-me-ters}
\hyphenation{sol-va-bi-li-ty}
%
% math code and graph stuff
\newtheorem{theorem}{Theorem}
\newtheorem{corollary}{Corollary}
\usepackage{listings}
\usepackage{minted}
\usepackage{tikz-qtree}
\usepackage{mdframed}
\usepackage[shortlabels]{enumitem}
\usepackage{pdfpages}
\usepackage{etoolbox}
% \usepackage{cite}
\renewcommand\listoflistingscaption{Source Code Listings}

\definecolor{mauve}{rgb}{0.58,0,0.82}
\definecolor{codegreen}{rgb}{0,0.6,0}
\definecolor{codegray}{rgb}{0.5,0.5,0.5}
\definecolor{codepurple}{rgb}{0.58,0,0.82}
\definecolor{backcolour}{rgb}{0.95,0.95,0.92}
\definecolor{LightGray}{gray}{0.9}

\lstset{frame=tb,
  language=C,
  backgroundcolor=\color{backcolour},
  commentstyle=\color{codegreen},
  keywordstyle=\color{blue},
  numberstyle=\tiny\color{codegray},
  % stringstyle=\color{mauve},
  stringstyle=\color{codepurple},
  % basicstyle=\ttfamily\footnotesize,
  basicstyle={\small\ttfamily},
  aboveskip=3mm,
  belowskip=3mm,
  columns=flexible,
  numbers=none,
  breaklines=true,
  keepspaces=true,
  showspaces=false,
  breakatwhitespace=true,
  showstringspaces=false,
  showtabs=false,
  tabsize=2,
  captionpos=b,
  % numbers=left,
  % numbersep=5pt
}

\newcounter{problem_counter}

%%%%%%%%%%%%%%%%%%%%%%%%%%%%%%%%%%%%%%%%%%%%%%%%%%%%%%%%%%%%%%%%%%%%%%%%%%%%%%%%%%%%%%%%%%%%%%%%%%%%%%%%%%%%%%%%%%%%%%%%%%%%%%%%%%%%%%%%

\renewcommand{\thesubsection}{\thesection.\alph{subsection}}
\newenvironment{problem}[2][Problem]
    % {\begin{document}
    { \begin{mdframed}[backgroundcolor=gray!15] \textbf{#1 #2} \\}
    {  \end{mdframed}}
    % {\textit{Solution:}}
    % {}
    % {\end{document}}
% Define solution environment
\newenvironment{solution}
    {\textit{Solution:}\\}
    {}

\renewcommand{\qed}{\quad\qedsymbol}
\tikzset{every tree node/.style={minimum width=0.5em,draw,circle},
         blank/.style={draw=none},
         edge from parent/.style=
         {draw,edge from parent path={(\tikzparentnode) -- (\tikzchildnode)}},
         level distance=1.5cm}

%%%%%%%%%%%%%%%%%%%%%%%%%%%%%%%%%%%%%%%%%%%%%%%%%%%%%%%%%%%%%%%%%%%%%%%%%%%%%%%%%%%%%%%%%%%%%%%%%%%%%%%%%%%%%%%%%%%%%%%%%%%%%%%%%%%%%%%%

\newcommand{\opus}[1]{%
    \begingroup
    \spaceskip=\fontdimen2\font plus \fontdimen3\font minus \fontdimen4\font
    \xspaceskip=\fontdimen7\font\relax
    \ttfamily
    %\hyphenchar\font=`\-
    #1%
    \endgroup
}

%
\begin{document}
%\ifpdf=\iftrue/

% since this a temple add label 'DONE' for the parts of the document already been written on top of the referring code
% for now, I'll incude them as 'TEMP' referring to template

\frontmatter
\pagenumbering{roman}

% recommended format (general use):
% - executive summary
% - introduction
    % - problem statement (whitepaper)
    % - scope and limitations
    % - audience and purpose of this presentation
% - development process (whitepaper - detailed documentation)
    % - Project management methodology (e.g., Agile, Waterfall)
    % - Project timeline and milestones
    % - Roles and responsibilities of team members
    % - Description of the development process and methodology used
    % - Tools and technologies used
    % - Legal and regulatory compliance
% - Requirements
    % - User requirements and expectations
    % - Functional requirements
    % - Non-functional requirements (e.g., performance, security, usability)
% - Design
    % - System architecture
    % - User interface design
    % - Data model and database design
    % - Class diagrams, sequence diagrams, and other relevant design documentation
% - methodology
% - technical description (whitepaper)
% - results
% - Implementation
    % - Code structure and organization
    % - Coding standards and guidelines
    % - Testing approach and results
    % - Bugs and issues encountered during development
% - use cases, market analysis (whitepaper)
    % - Unique selling proposition
    % - Industry overview and trends
    % - Target market and customer demographics
    % - Competitor analysis
    % - SWOT analysis
% User Manual
    % - Getting Started guide
    % - User interface overview
    % - User tasks and how to perform them
    % - Troubleshooting guide
    % - Frequently asked questions
% Maintenance and Support
    % - Ongoing maintenance and support plan
    % - Known issues and bug tracking
    % - Future enhancements and feature requests
% - discussion
% - conclusion
    % - Summary of the project and its achievements
    % - Lessons learned and recommendations for future projects
% - references
% - appendices
    % - Supporting documents such as resumes, legal documents, market research, and product specifications

% for things like certificate it is recommended to include the PDF of the signed document


%
\begin{titlingpage}
\begin{SingleSpace}
\calccentering{\unitlength}
\begin{adjustwidth*}{\unitlength}{-\unitlength}
%\vspace*{13mm}
\begin{center}
%\rule[0.5ex]{\linewidth}{2pt}\vspace*{-\baselineskip}\vspace*{3pt}
%\rule[0.5ex]{\linewidth}{1pt}\\[\baselineskip]
{\HUGE Building a Modern Hotel Management Application with Database System}\\[4mm]
\vspace{5mm}
% {\Large \textit{exploratory group project exercise for learning database concepts and implementation: entity-relationship modelling, role-based access control, data dictionaries, complex queries, working with forms \& reports}}\\
%\rule[0.5ex]{\linewidth}{1pt}\vspace*{-\baselineskip}\vspace{3pt}
%\rule[0.5ex]{\linewidth}{2pt}\\
% \vspace{3mm}
%\vspace{20mm}
{\large \textbf{Course ID.: CPE241 Database Systems}}\\
\vspace{3mm}
% This is our full version of project referencing to be used as a code in the database application
% [XXXX]X-0000-XX/X-X-000-XXX-000-X

%\vspace{54mm}

\includegraphics[scale=0.2]{logo/KMUTT_CI_Primary_Logo-Full-1200x1200.png}\\
\vspace{8mm}
{\large \textbf{Members:}}\\ % can hide this if it's only for single author
\vspace{3mm}
{\large{Chawit Pimapansri \\ ID: 65070503411 \texttt{<chawitpimapansri@gmail.com>}}} \\ 
{\large{Sorawit Tonpitak \\ ID: 65070503438 \texttt{<rocbot01@gmail.com>}}} \\ 
{\large{Nichaporn Manachaiprasert \\ ID: 65070503446 \texttt{<patty.nichapornn@gmail.com>}}} \\ 
{\large{Thanakit Chokbunsuwan \\ ID: 65070503448 \texttt{<tchokbunsuwan@gmail.com>}}} \\ 
{\large{Yuil Tripathee \\ ID: 65070503480 \texttt{<yuil.trip@mail.kmutt.ac.th>}}} \\ 
\vspace{3mm}

%{\large{Yuil Tripathee}}\\
%{\large{Student ID: 65070503480}}\\
\vspace{8mm}
{\large \textbf{Submitted To:} }\\
\vspace{3mm}
{\large {Department of Computer Engineering}}\\
{\large {in partial fulfillment of the requirements for the course of
CPE241 Database Systems.}}\\


\vspace{8mm}
{\large \textbf{Instructor:} }\\
\vspace{3mm}
{\large{Asst. Prof. Dr. Phond Phunchongharn}}\\
%{\large {Assistant Professor}}\\
{\large {Department of Computer Engineering}}\\
% \vspace{12mm}
% \includegraphics[scale=0.1]{frontmatter/BUET_LOGO_svg.png}\\
\vspace{8mm}

% if you have the institute name in the author's section then just leave with the date here
% \large {Part of DocSky Framework Project}\\
{\large KMUTT}\\
{\large May, 2024}\\
%\textsc
%\clearpage
\end{center}
%\begin{flushright}
%{\small Word count: ten thousand and four}
%\end{flushright}
\end{adjustwidth*}
\end{SingleSpace}
\end{titlingpage} % TEMP
% % Start of the revision history table
\begin{versionhistory}
      \vhEntry{v0.1}{2023-04-23}{Yuil Tripathee}{First release, include experimentation results}
       \vhEntry{v0.2}{2023-05-03}{Yuil Tripathee}{Added few coding scheme design reviews from small to large equipment manufacturers}
       \vhEntry{v1.0}{2023-05-07}{Yuil Tripathee}{Added more design review, methodology and conclusion to this white-paper}
   \end{versionhistory}

\clearpage % TEMP

\chapter*{Abstract}

We started this project to understand and apply database systems into a real world scenario. Hotel management system is one of the popular application projects being used in universities around the world to let students exercise their database design and implementation skills. We started with learning through the domain knowledge. However, things did not go as planned and we had to revise our ER model throughout our journey which we have documented as part of this report. Finally, a web application was developed (using Node.JS + MongoDB stack combination) and improvised so that multiple stakeholders can interact with our application. Each kind of end user has their own permission levels to different resources depending on the stake of their interaction with a hotel environment.

\textbf{Keywords}: database, hotel management, ER model

% EXAMPLE ABSTRACT:
% This project started with the objective of understanding database systems and applying the knowledge to practical usage. After consideration and research, we chose to create a digitized platform for human resources management. Human resources management is necessary for most companies and involves many processes that could be digitalized and automated. Therefore, it would be a good topic for our database design studies.

% ANSWER THE FOLLOWING QUESTIONS:
% What's going on?
% Why online hotel management?
% Work process?
% Development stacks!
% Challenges and findings along the way!
\clearpage
 % TEMP
% \chapter*{Acknowledgments}
%\begin{SingleSpace}
I am grateful for the University of Bristol members and the Bangladesh University of Engineering and Technology members. Thank you very much. We shall forever cherish the memories of working with him. We deeply thank our friends and families for always believing in us even at the moment when we were losing our confidence.
%\end{SingleSpace}
\clearpage % TEMP


\renewcommand{\contentsname}{Table of Contents}
\maxtocdepth{subsection}
\tableofcontents*
\addtocontents{toc}{\par\nobreak \mbox{}\hfill{\bf Page}\par\nobreak}
\clearpage

\listoftables
\addtocontents{lot}{\par\nobreak\textbf{{\scshape Table} \hfill Page}\par\nobreak}
%\clearpage

\listoffigures
\addtocontents{lof}{\par\nobreak\textbf{{\scshape Figure} \hfill Page}\par\nobreak}
\clearpage
%
% \lstlistoflistings
% \addcontentsline{toc}{chapter}{Code Listings}
% \addtocontents{lol}{\par\nobreak\textbf{{\scshape Code Listing} \hfill Page} \par\nobreak}
% \clearpage
%
%
% The bulk of the document is delegated to these chapter files in
% subdirectories.
\mainmatter
%
\import{chapters/chapter01-background/}{ch1-background.tex}
\import{chapters/chapter02-introduction/}{ch2-introduction.tex}
\import{chapters/chapter03-design/}{ch3-design.tex}
\import{chapters/chapter04-interface/}{ch4-UI.tex}
\import{chapters/chapter05-implement}{ch5-app_impl.tex}
\import{chapters/chapter06-findings}{ch5-findings.tex}
%\import{chapters/ch1-intro/}{ch1-intro.tex} % TEMP
%\import{chapters/ch2-design/}{ch2-design.tex} % TEMP
%\import{chapters/ch3-implement/}{ch3-apply.tex} % TEMP
%\import{chapters/ch4-record/}{ch4-record.tex} % TEMP

% other sections to include:
%  - literature review
%  - discussions

% topics related to methodologies can be separated if necessary (isolated context)
% example:
%   1. linear inverted pendulum system
%   2. rotary inverted pendulum system


%
\backmatter
%
% Add index
\printindex
\refstepcounter{chapter}
\renewcommand{\bibname}{References}
% unsrtnat for the order in text reference, plainnat for the order in author name
\bibliographystyle{unsrtnat}
\bibliography{bibliography.bib} % TEMP

% And the appendix goes here
\appendix
% \import{chapters/appendices/}{app0B.tex} % 
\import{chapters/appendices/}{app0A.tex}
%
\end{document}